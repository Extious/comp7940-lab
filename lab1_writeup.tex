\documentclass[11pt]{article}

\usepackage[margin=1in]{geometry}
\usepackage{graphicx}
\usepackage{hyperref}
\usepackage{xcolor}
\usepackage{listings}

\lstset{
  basicstyle=\ttfamily\small,
  breaklines=true,
  columns=fullflexible,
  frame=single,
  rulecolor=\color{black!20},
  keywordstyle=\color{blue!60!black},
  commentstyle=\color{black!60},
  stringstyle=\color{green!40!black},
  showstringspaces=false
}

\title{COMP7940 Cloud Computing\\Lab 1: Python Basic and GitHub Setup}
\author{
  Name: ZHAO Zhan \\
  Student ID: 25482351 \\
  GitHub: @Extious
}
\date{\today}

\begin{document}
\maketitle

\section*{1. Clone Command}
\begin{lstlisting}[language=bash]
git clone git@github.com:Extious/comp7940-lab.git
\end{lstlisting}

\section*{2. Screenshot of GitHub Repository with \texttt{hello.py}}
\medskip
\IfFileExists{lab1/github_hello.png}{
  \begin{center}
    \includegraphics[width=0.95\linewidth]{lab1/github_hello.png}
  \end{center}
}{
  \begin{center}
    \fbox{\parbox{0.95\linewidth}{
      Screenshot missing. Please add \texttt{lab1/github\_hello.png} and recompile.
    }}
  \end{center}
}

\section*{3. Why SSH Keys Are More Secure Than Passwords}
\begin{itemize}
  \item SSH uses public-key (asymmetric) cryptography: only the public key is uploaded; the private key stays on your computer \cite{rfc4251,rfc4252,github-about-ssh}.
  \item The private key is never sent over the network, which reduces the risk of credential interception \cite{rfc4252}.
  \item SSH keys can be protected with a local passphrase; even if the key file is copied, it is harder to use without the passphrase \cite{openssh-ssh-keygen}.
  \item Keys are resistant to password reuse and common phishing attacks because there is no typed password to steal \cite{github-about-ssh}.
  \item Keys can be revoked/rotated per device without changing other devices' access \cite{github-add-ssh-key}.
\end{itemize}

\section*{4. Python Code (Exercises 1--3)}

\subsection*{Exercise 1--3 Solution File}
\lstinputlisting[language=Python]{lab1/exercises.py}

\subsection*{\texttt{hello.py}}
\lstinputlisting[language=Python]{lab1/hello.py}

\bibliographystyle{plain}
\bibliography{lab1_writeup}

\end{document}
